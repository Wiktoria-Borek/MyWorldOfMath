\documentclass{article}
\usepackage{amsmath}
\hbadness=10000 % Suppress underfull hbox warnings

\title{Combinatorics}
\author{Wiktoria Borek}
\date{}

\begin{document}

\maketitle

\section{Basic Exercises}
\subsection{Exercise}
How many five-digit even numbers are there in which no digit repeats?
\subsubsection*{Solution}
To form a five-digit even number, the last digit must be one of the even digits.\\
Therefore we have the following choices for the last digit: 0, 2, 4, 6, 8.\\
\\
There are \textbf{5 choices} for the last digit.\\
However, we must consider the cases where the last digit is 0 and where it is not.\\
\\
That's because if we choose 0 as the last digit, then the choice doesn't affect the first digit,
but if we choose any other even digit, then it reduces the choices for the first digit with one less option.\\
\\
We split the cases as follows:\\
$\overset{}{\overline{9}} \cdot \overset{}{\overline{8}} \cdot \overset{}{\overline{7}} \cdot \overset{}{\overline{6}} \cdot \overset{0}{\overline{1}} + \overset{}{\overline{8}} \cdot \overset{}{\overline{8}} \cdot \overset{}{\overline{7}} \cdot \overset{}{\overline{6}} \cdot \overset{2,4,6,8}{\overline{4}}$
\subsection{Exercise}
How many different 4-digit safe codes are there with exactly one odd digit in them?
\subsubsection*{Solution}
First of all, the provided description of the problem tells us that's safe codes\\
and not numbers, so we can use 0 as the first digit.\\
\\
What's more, the problem states that there is exactly one odd digit\\
in the code, which means that the other three digits must be even.\\
\\
There are 5 odd and also 5 even digits.
\\
We can choose the position of the odd digit in 4 different ways,\\
as the code has 4 digits.\\
\\
Therefore, the total number of different 4-digit safe codes with exactly one odd digit is given by:\\
$(4 \cdot \overset{odd}{\overline{5}}) \cdot \overset{even}{\overline{5}} \cdot \overset{even}{\overline{5}} \cdot \overset{even}{\overline{5}} = 4 \cdot 5^4$
\subsection{Exercise}
How many different 4-digit safe codes are there in which there is exactly one odd digit and all the digits are different?
\subsubsection*{Solution}
The situation is similar to the previous exercise,\\
but now we have the restriction that all digits must be \textbf{different}.\\
$(4 \cdot \overset{odd}{\overline{5}}) \cdot \overset{even}{\overline{5}} \cdot \overset{even}{\overline{4}} \cdot \overset{even}{\overline{3}} = 5^2 \cdot 4^2 \cdot 3$
\subsection{Exercise}
Bill has in his closet pants in 4 colors: black, blue, green and red; shirts in 3 colors: white, blue and green;
and hats in 5 different colors. How many different sets of clothes consisting of 1 pants, 1 shirt and 1 hat can Bill create?
How many such sets are there in which the pants and shirt are one color?
\subsubsection*{Solution}
Let's begin answering the first question.\\
This question itself explains that Bill has 4 pants, 3 shirts and 5 hats,\\
because it treats them as separate items of one color each.\\
\\
Therefore, the total number of different sets of clothes consisting of 1 pants, 1 shirt and 1 hat is given by:\\
$4 \cdot 3 \cdot 5 = 60$\\
\\
Now let's answer the second question.\\
\\
We find that the pants and shirt are one color, where the color is blue or green,\\
so there are 2 colors that can be chosen for both pants and shirt.\\
For both pairs of one color each, there are 5 choices for the hat,\\
because the hat can be of any color being mentioned previously,\\
but not to be a specific color.\\
\\
Therefore, the total number of sets of clothes in which the pants and shirt are one color is given by:\\
$2 \cdot 5 = 10$\\
\\
That's because for 2 sets we take 5 hats separately, where each set has\\
pants and shirt of the same color.
\end{document}
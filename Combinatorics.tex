\documentclass{article}
\usepackage{amsmath}
\hbadness=10000 % Suppress underfull hbox warnings

\title{Combinatorics}
\author{Wiktoria Borek}
\date{}

\begin{document}

\maketitle

\section{Basic Exercises}
\subsection{Exercise}
How many five-digit even numbers are there in which no digit repeats?
\subsubsection*{Solution}
To form a five-digit even number, the last digit must be one of the even digits.\\
Therefore we have the following choices for the last digit: 0, 2, 4, 6, 8.\\
\\
There are \textbf{5 choices} for the last digit.\\
However, we must consider the cases where the last digit is 0 and where it is not.

That's because if we choose 0 as the last digit, then the choice doesn't affect the first digit,
but if we choose any other even digit, then it reduces the choices for the first digit with one less option. \\
\\
We split the cases as follows:\\

$\overset{}{\overline{9}} \cdot \overset{}{\overline{8}} \cdot \overset{}{\overline{7}} \cdot \overset{}{\overline{6}} \cdot \overset{0}{\overline{1}} + \overset{}{\overline{8}} \cdot \overset{}{\overline{8}} \cdot \overset{}{\overline{7}} \cdot \overset{}{\overline{6}} \cdot \overset{2,4,6,8}{\overline{4}}$
\end{document}